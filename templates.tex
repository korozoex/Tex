\documentclass{jsarticle}


\title{電気系特別講義第二}
\author{機械工学科 3年 渡部拓哉}
\date{\today}




\begin{document}
\maketitle

\section{第一回 10/3}
%\negin{align}
講師:有田政史講師\\
議題細目:鉄鋼業における電気・情報。電子系技術の応用展開\\
%\end{align}
7回の出席、出席は毎回の授業のレポート。\\





\section{見出し}

この文書の先頭にはタイトル,著者名,日付が出力されています。
特定の日付を指定することもできます。

そして,セクションの見出しが出力されています。
セクションの番号は自動的に付きます。

\section{箇条書き}

以下は箇条書きの例です。これは番号を振らない箇条書きです。

\begin{itemize}
  \item りぼん
  \item なかよし
\end{itemize}

これは番号を振る箇条書きです。

\begin{enumerate}
  \item 富士
  \item 鷹
  \item なすび
\end{enumerate}

\section{おわりに}

これは一段組の例ですが,二段組に変更することもできます。

解説文を読んで,このソースをいろいろと変更してみましょう。

\end{document}